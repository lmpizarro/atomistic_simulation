
\section{Descripci\'on del sistema de c\'alculo}


Se desarrollo un c\'odigo de c\'alculo en Fortran y dos c\'odigos
en python uno  de pre y el otro de post-procesamiento.


\subsection{C\'odigo ising}

\subsubsection{Par\'ametros de entrada}



\paragraph{Pre procesamiento en python}
En la carpeta src/ el archivo parametros.py contiene la parametrizaci\'on 
de los scripts
en python que permiten correr el sistema en serie (ver \ref{serie}), 
o en paralelo (ver \ref{paralelo}).


\subsubsection{Salida del programa}

\subsubsection{Librer\'ias utilizadas}

La carpeta libs/ contiene las librer\'ias usadas por el sistema

\paragraph{estadistica.f90}

Un conjunto de subrutinas para c\'alculo estad\'istico. 

\paragraph{globales.f90}
Un m\'odulo que contiene las variables globales del programa.
	
\paragraph{io\_parametros.f90}  
Un m\'odulo para el manejo de la entrada y las salidas de los
datos del programa


\paragraph{isingmods.f90} 
En este m\'odulo  se encuentran las principales rutinas de c\'alculo del
sistema.


\paragraph{strings.f90}
Un m\'odulo que contiene subrutinas de manejo de cadena de caracteres.


\paragraph{usozig.f90} 
En este m\'odulo se ecuentran las subrutinas que usan ziggurat. 
					
					
\paragraph{ziggurat.f90}

Para la generaci\'on de n\'umeros aleatorios se utilizaron
las capacidades provistas por la librer\'ia ziggurat. 



\subsection{Corridas en serie}\label{serie}

\subsection{Corridas en paralelo}\label{paralelo}
