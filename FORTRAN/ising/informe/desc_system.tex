\section{Descripci\'on del sistema de c\'alculo}


Se desarrollo un c\'odigo de c\'alculo en Fortran y dos c\'odigos
en python uno  de pre y el otro de post-procesamiento. Tanto el código de preprocesamiento, 
como el código de postprocesamiento, comprenden varios módulos responsables de distintas
tareas.

El c\'odigo fuente del programa \textbf{\textit{ising.f90}}, se encuentra en la carpeta
\textbf{src/}, su compilación se hace con el sistema \textbf{cmake} y depende
de un conjunto de módulos que se encuentra en la carpeta \textbf{libs}.

Para compilarlo hay que crear hacer:

\begin{verbatim}
  > mkdir build/
  > cd build/
  > cmake ..
\end{verbatim}

esas operaciones crean un directorio \textbf{bin/} donde se encuentran todos
los elementos necesarios para hacer las corridas del programa.
Para eso hay que parametrizar adecuadamente el sistema en \textbf{\textit{parametros.py}}.

Los comentarios del módulo explican el fin de cada una de las variables globales python.

\begin{verbatim}
# -*- coding: utf-8 -*-

###############################################################################       
#   PARAMETROS DE ENTRADA
###############################################################################
# Tamaño de la red de spines
N_red = 20
M_red = 20
# Cada cuántos puntos se quiere grabar el archivo temporal
# K_grab = 0 especifica que no se grabe ningún archivo temporal
N_grab = 0     
# Barrido de temperaturas
# Temperatura mínima
T_min = 0.5
# Temperatura máxima
T_max = 4.0
# Paso de temperatura
dT = 0.1
# Agrego el detalle cerca de la temperatura crítica
T_detail_min = 2.10
T_detail_max = 2.50
dT_detail = 0.02
# Número de pasos para la primer corrida (termalización)
N_term = '4000'
# Número de pasos para la segunda corrida (medición)
N_medi = '10000'
# Número de corridas para cada temperatura
Nrun = 8
#
# FIN PARAMETROS DE ENTRADA
###############################################################################

\end{verbatim}


y luego hacer, para correr la versión serie:

\begin{verbatim}
  > cd bin/
  > python corridas.py
\end{verbatim}

o si se quiere correr la versión paralelo:

\begin{verbatim}
  > cd bin/
  > python corridas_paralelos.py
\end{verbatim}


Una vez que se terminan los cálculos para todas las temperaturas,  en 
la carpeta \textbf{bin/}, se corre:

\begin{verbatim}
  > python grafico_tablas.py 
\end{verbatim}

para obtener los gráficos de las medidas fundamentales del sistema.

Si se quieren obtener los gráficos de las correlaciones a distintas temperaturas, 
se debe correr:

\begin{verbatim}
  > python correlaciones.py 
\end{verbatim}



\subsection{C\'odigo ising}

\subsubsection{Par\'ametros de entrada}

El programa ising lee la configuraci\'on de una corrida del 
archivo parametros.dat que tiene que estar en la carpeta donde se
corre ising.

El archivo \textbf{parametros.dat} tiene la forma: 

\begin{verbatim}
     N M T J MCS PE. 
\end{verbatim}

Donde NxM 
son las dimensiones de la matriz, T la temperatura de la corrida,
J el par\'ametro de c\'alculo de energ\'ia de Ising, MCS son los pasos Monte Carlo y PE
es una cantidad que especifica cada cuantos pasos Monte Carlo se graba la energ\'ia.

\begin{verbatim}
   20 20 2.22 1 500 100
\end{verbatim}


\paragraph{Preprocesamiento en python}
En la carpeta \textbf{src/} el archivo parametros.py contiene la parametrizaci\'on 
de los scripts
en python que permiten correr el sistema en serie (ver \ref{serie}), 
o en paralelo (ver \ref{paralelo}).
El preprocesamiento en python, permite definir una zona de c\'alculo de paso
de temperatura fino. Es decir se puede parametrizar un paso de temperatura de
0,1 para todo el rango de temperatura y dentro de ese rango definir una zona
donde el paso de temperatura puede ser menor, ej. $\Delta T$ = 0.01. Para lograr una mejor resoluci\'on
en esa zona.

En particular en este c\'alculo se uso para explorar con m\'as detalle la zona
cr\'itica.

El preprocesamiento en python, crea el archivo parametros.dat, 
 corre el programa de ising a distintas temperaturas
  Crea una carpeta para cada temperatura. En cada una de esas carpetas, a su
  vez, crea Nruns carpetas para hacer estadística y obtener los valores con
  sus respectivos errores.
 
  En cada temperatura, se utiliza el valor final de la temperatura anterior.
  Arbitrariamente, se toma el valor final de RUN00 como el valor inicial de
  todas las corridas a la siguiente temperatura (menor)
 

\subsubsection{Salida del programa}

Los resultados los guarda en el archivo tablas\_temperatura.dat.

Cada fila de tablas\_temperatura.dat representa los resultados a una dada
temperatura con sus respectivos errores. Las columnas son:

\begin{verbatim}
  T <E> std(E) <M> std(M) <c> std(c) <suc> std(suc) <acept> std(acept)
\end{verbatim}

Donde:



\begin{itemize}
  \item T: temperatura de la corrida 
    \item <E>: Energ\'ia media calculada
    \item  std(E): desviaci\'on standard de la energ\'ia calculada

    \item <M>: Magnetizaci\'on media calculada
    \item  std(M): desviaci\'on standard de la Magnetizaci\'on calculada


    \item <c>: Capacidad calor\'ifica media calculada
    \item  std(c): desviaci\'on standard de la Capacidad calor\'ifica calculada

    \item <suc>: Susceptibilidad magn\'etica media calculada
    \item  std(suc): desviaci\'on standard de la Susceptibilidad magn\'etica calculada


    \item <acept>: Aceptaci\'on media calculada
    \item  std(acept): desviaci\'on standard de la Aceptaci\'on calculada
\end{itemize}



  Los errores se calculan como std(RUN)/sqrt(NRUN)


\subsubsection{Librer\'ias utilizadas}

La carpeta \textbf{libs/} contiene las librer\'ias usadas por el sistema

\paragraph{\underline{\textit{estadistica.f90}}}

Un conjunto de subrutinas para c\'alculo estad\'istico. 

\paragraph{\underline{\textit{globales.f90}}}
Un m\'odulo que contiene las variables globales del programa.
	
\paragraph{\underline{\textit{io\_parametros.f90}}}  
Un m\'odulo para el manejo de la entrada y las salidas de los
datos del programa.


\paragraph{\underline{\textit{isingmods.f90}}} 
En este m\'odulo  se encuentran las principales rutinas de c\'alculo del
sistema.


\paragraph{\underline{\textit{strings.f90}}}
Un m\'odulo que contiene subrutinas de manejo de cadena de caracteres.


\paragraph{\underline{\textit{usozig.f90}}} 
En este m\'odulo se ecuentran las subrutinas que usan ziggurat y que
son usadas en el programa principal, ising.
					
					
\paragraph{\underline{\textit{ziggurat.f90}}}

Para la generaci\'on de n\'umeros aleatorios se utilizaron
las capacidades provistas por la librer\'ia ziggurat. 



\subsection{Corridas en serie}\label{serie}

\subsection{Corridas en paralelo}\label{paralelo}
