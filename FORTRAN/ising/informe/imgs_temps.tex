
\section {Evoluci\'on de la matriz de estado de spines en funci\'on de la temperatura}


Al final de cada ciclo de cálculo en cada una de las  temperaturas, se obtiene
la matriz de estado de los spines del sistema.
Las figuras \ref{fig:Talta}, \ref{fig:Tmedia},  \ref{fig:Tbaja},  
%\ref{fig:frame2.0},  \ref{fig:frame1.3},  \ref{fig:frame0.5}, 
muestran el estado
del sistema a las temperaturas: 5.0, 3.8, 2.3, 2.0, 1.3 y 0.5 respectivamente.


\begin{figure}[H]
	\centering
\begin{subfigure}{.49\textwidth}
	\centering
      \includegraphics[scale=0.3]{{frame5.0}.pdf} \\
      \caption{Temperatura 5.0}\label{fig:frame5.0}
\end{subfigure}
\begin{subfigure}{.49\textwidth}
	\centering
      \includegraphics[scale=0.3]{{frame3.8}.pdf} \\
      \caption{Temperatura 3.8}\label{fig:frame3.8}
\end{subfigure}
      \caption{Temperaturas altas}\label{fig:Talta}
\end{figure}

\begin{figure}[H]
	\centering
\begin{subfigure}{.49\textwidth}
	\centering
      \includegraphics[scale=0.3]{{frame2.3}.pdf} \\
      \caption{Temperatura 2.3}\label{fig:frame2.3}
\end{subfigure}
\begin{subfigure}{.49\textwidth}
	\centering
      \includegraphics[scale=0.3]{{frame2.0}.pdf} \\
      \caption{Temperatura 2.0}\label{fig:frame2.0}
\end{subfigure}

\caption{Temperaturas medias}\label{fig:Tmedia}
\end{figure}

\begin{figure}[H]
	\centering
\begin{subfigure}{.49\textwidth}
	\centering
      \includegraphics[scale=0.3]{{frame1.3}.pdf} \\
      \caption{Temperatura 1.3}\label{fig:frame1.3}
\end{subfigure}
\begin{subfigure}{.49\textwidth}
	\centering
      \includegraphics[scale=0.3]{{frame0.5}.pdf} \\
      \caption{Temperatura 0.5}\label{fig:frame0.5}
\end{subfigure}

\caption{Temperaturas bajas}\label{fig:Tbaja}
\end{figure}


