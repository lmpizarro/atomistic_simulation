\section {Evoluci\'on de la matriz de estado de spines en funci\'on de la 
temperatura}


Al final de cada ciclo de c\'alculo, en cada una de las  temperaturas, se 
obtiene
la matriz de estado de los spines del sistema.

Un script en python recorre cada una de las carpetas, lee el archivo 
ultimo\_estado.txt,
que graba el programa ising, lo convierte a un array 2D de numpy, para luego 
con la 
librer\'ia matplotlib convertirlo a imagen y grabarlo en disco. 
Luego de grabado en disco con el programa ffmpeg, se obtiene un video que 
muestra
la secuencia de la evoluci\'on del sistema desde temperaturas altas a bajas.



Las figuras \ref{fig:Talta}, \ref{fig:Tmedia},  \ref{fig:Tbaja},  
%\ref{fig:frame2.0},  \ref{fig:frame1.3},  \ref{fig:frame0.5}, 
muestran algunas im\'agenes de esa secuencia correspondientes a  estados
del sistema a temperaturas altas, medias y bajas respectivamente, 
T: 5.0, 3.8, 2.3, 2.0, 1.3 y 0.5.

Al disminuir la temperatura se observa la tendencia a que un valor de spin
tenga m\'as presencia sobre la matriz que el otro. En temperaturas bajas el
spin que mostraba tendencia mayoritaria a temperaturas m\'as altas, termina
siendo el único spin presente en la matriz.




\begin{figure}[H]
	\centering
\begin{subfigure}{.49\textwidth}
	\centering
      \includegraphics[scale=0.3]{{frame5.0}.pdf} \\
      \caption{Temperatura 5.0}\label{fig:frame5.0}
\end{subfigure}
\begin{subfigure}{.49\textwidth}
	\centering
      \includegraphics[scale=0.3]{{frame3.8}.pdf} \\
      \caption{Temperatura 3.8}\label{fig:frame3.8}
\end{subfigure}
      \caption{Temperaturas altas}\label{fig:Talta}
\end{figure}

\begin{figure}[H]
	\centering
\begin{subfigure}{.49\textwidth}
	\centering
      \includegraphics[scale=0.3]{{frame2.3}.pdf} \\
      \caption{Temperatura 2.3}\label{fig:frame2.3}
\end{subfigure}
\begin{subfigure}{.49\textwidth}
	\centering
      \includegraphics[scale=0.3]{{frame2.0}.pdf} \\
      \caption{Temperatura 2.0}\label{fig:frame2.0}
\end{subfigure}

\caption{Temperaturas medias}\label{fig:Tmedia}
\end{figure}

\begin{figure}[H]
	\centering
\begin{subfigure}{.49\textwidth}
	\centering
      \includegraphics[scale=0.3]{{frame1.3}.pdf} \\
      \caption{Temperatura 1.3}\label{fig:frame1.3}
\end{subfigure}
\begin{subfigure}{.49\textwidth}
	\centering
      \includegraphics[scale=0.3]{{frame0.5}.pdf} \\
      \caption{Temperatura 0.5}\label{fig:frame0.5}
\end{subfigure}

\caption{Temperaturas bajas}\label{fig:Tbaja}
\end{figure}
