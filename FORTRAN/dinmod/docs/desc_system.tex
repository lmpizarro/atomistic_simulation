

\section{Descripci\'on del sistema de c\'alculo}


Se desarrollo un c\'odigo de c\'alculo en Fortran y dos c\'odigos
en python uno  de pre y el otro de postprocesamiento. Tanto el código de 
preprocesamiento, 
como el código de postprocesamiento, comprenden varios módulos responsables de distintas
tareas.

El proyecto se encuentra bajo control de versión en un repositorio \textbf{git} de CNEA
y se obtiene con los permisos adecuados haciendo:

\begin{verbatim}
> git clone https://lpizarro@scmmanager.dcap.cnea.gov.ar/scm/git/SimRui
\end{verbatim}


Una vez cloneado el repositorio, para acceder al código, hacer:

\begin{verbatim}
  > cd FORTRAN/dinmod
\end{verbatim}



El c\'odigo fuente del programa \textbf{\textit{dinmod.f90}}, se encuentra en la carpeta
\textbf{src/}, su compilación se hace con el sistema \textbf{cmake} y depende
de un conjunto de módulos que se encuentra en la carpeta \textbf{libs}.

Para compilarlo hay que corre el script instalar.sh:

\begin{verbatim}
  > ./instalar.sh 
\end{verbatim}

esas operaciones crean un directorio  \textbf{build/bin/} donde se encuentran todos
los elementos necesarios para hacer las corridas del programa.

El control de las corridas del código en FORTRAN se hace desde scripts en el lenguaje
de alto nivel phyton \href{http://www.python.org/}. 

Para eso hay que parametrizar adecuadamente el sistema en \textbf{\textit{parametros_t.py}}
\textbf{\textit{parametros_rho.py}}, en un caso se parametriza la dependecia en temperaturas
de las corridas y en el otro la dependencia en la densidad.
 
Los comentarios del módulo explican el fin de cada una de las variables globales python.

\begin{verbatim}

###############################################################################       
#   PARAMETROS DE ENTRADA PARA CORRER A DISTINTAS TEMPERATURAS
###############################################################################
# Número de partículas
N_part = 200
# Densidad de partículas
Rho = 0.3
#------ Barrido de temperaturas
# Temperatura mínima
Temp_min = 0.7
# Temperatura máxima
Temp_max = 1.4
# Paso de temperatura
dTemp = 0.25
# Agrego el detalle cerca de la temperatura crítica
#T_detail_min = 2.10
#T_detail_max = 2.50
#dT_detail = 0.02
# abs(K_grab) Cada cuántos puntos se quiere grabar el archivo temporal
# K_grab < 0 especifica que no se grabe ningún archivo temporal
N_grab = 10
# Paso temporal de integración
dt = 0.001
# Número de pasos para la primer corrida (termalización)
N_term = 1000
# Número de pasos para la segunda corrida (medición)
N_medi = 2000
# Número de corridas para cada temperatura
Nrun = 8
#
# FIN PARAMETROS DE ENTRADA
###############################################################################

\end{verbatim}


y luego hacer, para correr la versión serie:

\begin{verbatim}
  > cd bin/
  > python corridas.py
\end{verbatim}

o si se quiere correr la versión paralelo:

\begin{verbatim}
  > cd bin/
  > mpirun.mpich -n 8 python corridas_paralelos.py
\end{verbatim}

Luego de una corrida en una serie de temperaturas, queda una estructura de directorios
como la que se muestra en \eqref{fig:arboldir}, con la información necesaria 
para postprocesar
los resultados y obtener los valores de las mediciones en función de la temperatura.


\tikzstyle{every node}=[thick,anchor=west, 
font={\scriptsize\ttfamily}, inner sep=2.5pt]
\tikzstyle{selected}=[draw=blue,fill=blue!10]
\tikzstyle{root}=[draw=blue, fill=blue!30]

\begin{figure}[H]
\begin{center}
\begin{tikzpicture}[%
    scale=.7,
    grow via three points={one child at (0.5,-0.65) and
    two children at (0.5,-0.65) and (0.5,-1.2)},
    edge from parent path={(\tikzparentnode.south) |- (\tikzchildnode.west)}]
  	\node [root] {bin}
  	child { node [red] {ising}}
  	child { node [red] {corridas\_paralelo.py}}
  	child { node [red] {corridas.py}}
  	child { node {tablas\_temperatura.dat}}
    child { node [selected] {0.5\_tmpfolder}
      child { node {parametros.dat}}
      child { node {runs\_estadistica.dat}}
      child { node [selected] {RUN00}
	     child { node {val\_medios.dat}}
	     child { node {estado.dat}}
	     child { node {ultimo\_estado.dat}}
      }
      child { node at(0,-1.8) [selected] {RUN01}}
      child { node at(0,-1.9) [selected] {RUN02}}
      child { node at (0,-2.0) {\dots}}
    }       
    child { node at (0,-5.5) [selected]{0.6\_tmpfolder}}
    child { node at (0,-5.8)[selected] {0.7\_tmpfolder}}
    child { node at (0,-6.1) {\dots}};
\end{tikzpicture}
\end{center}
\caption{Esquema de carpetas y archivos que se obtienen al realizar una corrida 
completa haciendo estadísticas en cada temperatura.}
\label{fig:arboldir}
\end{figure}

Una vez que se terminan los cálculos para todas las temperaturas,  en 
la carpeta \textbf{bin/}, se corre:

\begin{verbatim}
  > python grafico_tablas.py 
\end{verbatim}

para obtener los gráficos de las medidas fundamentales del sistema.

Si se quieren obtener los gráficos de las correlaciones a distintas temperaturas, 
se debe correr:

\begin{verbatim}
  > python correlaciones.py 
\end{verbatim}



\subsection{C\'odigo ising}

\subsubsection{Par\'ametros de entrada}

El programa ising lee la configuraci\'on de una corrida del 
archivo \texttt{parametros.dat} que tiene que estar en la carpeta donde se
corre ising.

El archivo \texttt{parametros.dat} tiene la forma: 

\begin{verbatim}
     N M T J MCS PE. 
\end{verbatim}

Donde NxM 
son las dimensiones de la matriz, T la temperatura de la corrida,
J el par\'ametro de c\'alculo de energ\'ia de Ising, MCS son los pasos Monte Carlo y PE
es una cantidad que especifica cada cuantos pasos Monte Carlo se graba la energ\'ia.

\begin{verbatim}
   20 20 2.22 1 500 100
\end{verbatim}


\paragraph{Preprocesamiento en python}
En la carpeta \textbf{src/} el archivo \texttt{parametros.py} contiene la 
parametrizaci\'on de los scripts en python que permiten correr el sistema en 
serie (ver \eqref{serie}), 
o en paralelo (ver \eqref{paralelo}).
El preprocesamiento en python, permite definir una zona de c\'alculo de paso
de temperatura fino. Es decir se puede parametrizar un paso de temperatura de
$0.1$ para todo el rango de temperatura y dentro de ese rango definir una zona
donde el paso de temperatura puede ser menor, ej. $\Delta T = 0.02$. Para 
lograr una mejor resoluci\'on
en esa zona.

En particular en este c\'alculo se uso para explorar con m\'as detalle la zona
cr\'itica.

El preprocesamiento en python, crea el archivo \texttt{parametros.dat}, 
 corre el programa de ising a distintas temperaturas y crea una carpeta para 
 cada temperatura. En cada una de esas carpetas, a su vez, crea Nruns carpetas 
 para hacer estadística y obtener los valores con sus respectivos errores.
 
  En cada temperatura, se utiliza el valor final de la temperatura anterior.
  Arbitrariamente, se toma el valor final de RUN00 como el valor inicial de
  todas las corridas a la siguiente temperatura (menor).
 

\subsubsection{Salida del programa}

Los resultados los guarda en el archivo \texttt{tablas\_temperatura.dat}.

Cada fila de \texttt{tablas\_temperatura.dat} representa los resultados a una 
dada temperatura con sus respectivos errores. Las columnas son:

\begin{verbatim}
  T <E> std(E) <M> std(M) <c> std(c) <suc> std(suc) <acept> std(acept)
\end{verbatim}

Donde:



\begin{itemize}
  \item $T$: temperatura de la corrida 
    \item $<E>$: Energ\'ia media calculada
    \item  $std(E)$: desviaci\'on standard de la energ\'ia calculada

    \item $<M>$: Magnetizaci\'on media calculada
    \item  $std(M)$: desviaci\'on standard de la Magnetizaci\'on calculada


    \item $<c>$: Capacidad calor\'ifica media calculada
    \item  $std(c)$: desviaci\'on standard de la Capacidad calor\'ifica calculada

    \item $<suc>$: Susceptibilidad magn\'etica media calculada
    \item  $std(suc)$: desviaci\'on standard de la Susceptibilidad magn\'etica calculada


    \item $<acept>$: Aceptaci\'on media calculada
    \item  $std(acept)$: desviaci\'on standard de la Aceptaci\'on calculada
\end{itemize}


\subsubsection{Librer\'ias utilizadas}

La carpeta \textbf{libs/} contiene las librer\'ias usadas por el sistema

\paragraph{\underline{\textit{estadistica.f90}}}

Un conjunto de subrutinas para c\'alculo estad\'istico. 

\paragraph{\underline{\textit{globales.f90}}}
Un m\'odulo que contiene las variables globales del programa.
	
\paragraph{\underline{\textit{io\_parametros.f90}}}  
Un m\'odulo para el manejo de la entrada y las salidas de los
datos del programa.


\paragraph{\underline{\textit{isingmods.f90}}} 
En este m\'odulo  se encuentran las principales rutinas de c\'alculo del
sistema.


\paragraph{\underline{\textit{strings.f90}}}
Un m\'odulo que contiene subrutinas de manejo de cadena de caracteres.


\paragraph{\underline{\textit{usozig.f90}}} 
En este m\'odulo se ecuentran las subrutinas que usan ziggurat y que
son usadas en el programa principal, ising.
					
					
\paragraph{\underline{\textit{ziggurat.f90}}}

Para la generaci\'on de n\'umeros aleatorios se utilizaron
las capacidades provistas por la librer\'ia ziggurat. 


\subsection{Corridas en serie}\label{serie}
El módulo python \texttt{corridas.py}, es el que tiene la capacidad de correr
el código de cálculo en una máquina con un único núcleo. Parametrizando 
adecuadamente \texttt{parametros.py} se logran tiempos de corrida razonables en 
una máquina con un único procesador. El parámetro que permite acelerar la 
velocidad de procesamiento, es el que indica cada cuantos pasos Monte Carlo se 
quieren grabar las energías.

\subsection{Corridas en paralelo}\label{paralelo}

El módulo python \texttt{corridas\_paralelos.py}, es el que tiene la capacidad 
de correr en una máquina con varios núcleos.

El código se paraleliza usando el paquete \href{http://mpi4py.scipy.org/} {\textbf{mpi4py}} ,  desarrollado por Lisandro Dalcin.
De esa manera lo que se hizo, es paralelizar desde python y no desde Fortran.

El criterio utilizado para paralelizar es hacer en simultaneo  las corridas a 
una dada temperatura. 
Una vez que terminan los cálculos a una dada temperatura, se hacen los cálculos 
en la próxima temperatura con el estado final de la temperatura anterior.

Este diseño permitió hacer cálculos en sistemas de mayor tamaño, lograndose llegar a dimensiones
de 100x100. De esa manera se obtuvieron las curvas que muestran el comportamiento del sistema en
tamaños mayores a 20x20.
