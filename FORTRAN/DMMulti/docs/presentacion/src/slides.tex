%\documentclass[table,handout]{beamer}
\documentclass[hyperref={pdfpagemode=FullScreen},table]{beamer}
\usepackage[latin1]{inputenc}
\usepackage[spanish,es-nodecimaldot]{babel}
\usepackage{amsmath}
\usepackage{amsfonts}
\usepackage{colortbl}
\usepackage{multirow,caption}
%\usepackage{booktabs}
\usepackage{tabularx}
%\usepackage{amssymb}
\usepackage{appendixnumberbeamer} % corrige las p�ginas en el ap�ndice

\usepackage{tikz}
\usetikzlibrary{shapes,snakes,shadows,arrows,calc,decorations.markings}
\usepackage{pgfplots}
\pgfplotsset{compat=1.9}

%\usepackage{beamerthemesplit}

%% Distintos temas disponibles
%\usetheme{Frankfurt} % bien
%\usetheme{Bergen}
%\usetheme{Berlin} % bien
%\usetheme{Copenhagen}
%\usetheme{Berkeley} % maso
\usetheme{Darmstadt} % mejor
%\usetheme{Dresden}
%\usetheme{Warsaw}
%\usetheme{AnnArbor} %feo
%\usetheme{Montpellier}
%\usetheme{Madrid} %bueno

\useoutertheme[subsection=false,footline=title]{miniframes}

\graphicspath{{Figuras/}}      %Para tener los gr�ficos en una carpeta

%% Cambia los colores de un tema
%\usecolortheme[]{crane}
%\usecolortheme{sidebartab}
\usecolortheme[RGB={40,100,50}]{structure}
\beamertemplatesolidbackgroundcolor{green!2}

%% Aparece sombreado los items no vistos
\beamersetuncovermixins{\opaqueness<1->{20}}{\opaqueness<2->{15}}

%% Uso de la fuente
%\usefonttheme[onlylarge]{structurebold}

%% Barra lateral
%\setbeamersize{sidebar width left=1cm}

%\setbeamertemplate{sidebar canvas left}[horizontal shading]

%% Cambia tick box en los iconos de la navegaci�n
\setbeamertemplate{mini frames}[default] %default - box - tick

%% Quita la barra navegadora por defecto
\setbeamertemplate{navigation symbols}{}

%% Hace que la bibiolgrafia sea un librito
%\setbeamertemplate{bibliography item}[book]

%% Cambia el color de alerta, en bloque y en palabras.
\setbeamercolor{alerted text}{fg=orange!70!black}

%%%%%%%%%%%%%%%%%%%%%%%%%%%%%%%%%%%%%%%%%%%%%%%%%%%%%%%%%%%%%%%%%%%%
%% Define la barra de arriba

%\setbeamercolor{postit}{bg=blue!70}

\setbeamertemplate{headline}
{%
%\begin{beamercolorbox{postit}
\begin{beamercolorbox}{section in head/foot}
\vskip2pt\insertnavigation{\paperwidth}\vskip4pt
%\insertpagenumber
\end{beamercolorbox}%
}

%%%%%%%%%%%%%%%%%%%%%%%%%%%%%%%%%%%%%%%%%%%%%%%%%%%%%%%%%%%%%%%%%%%%
%%% Define la barra de abajo
%\setbeamertemplate{footline}
%{%
%\begin{beamercolorbox}{section in head/foot}
%\vskip2pt\hskip4pt\insertshorttitle[width=10cm]\hskip2cm\insertframenumber\vskip2pt
%%\insertframenumber
%\end{beamercolorbox}%
%}

%% Est� copiado del outertheme split
\setbeamertemplate{footline}
{%
  \hbox{\begin{beamercolorbox}[wd=.5\paperwidth,ht=2.5ex,dp=1.125ex,leftskip=.3cm plus1fill,rightskip=.3cm]{author in head/foot}%
    \usebeamerfont{author in head/foot}\inserttitle
  \end{beamercolorbox}%
  \begin{beamercolorbox}[wd=.5\paperwidth,ht=2.5ex,dp=1.125ex,leftskip=.3cm,rightskip=.3cm plus1fil]{title in head/foot}%
    \usebeamerfont{title in head/foot}
    \flushright{ \insertframenumber \, de \inserttotalframenumber
    \hskip0.1cm}
  \end{beamercolorbox}}%
  \vskip0pt%
}
%\setbeamertemplate{footline}[frame number]

%%%%%%%%%%%%%%%%%%%%%%%%%%%%%%%%%%%%%%%%%%%%%%%%%%%%%%%%%%%%%%%%%%%%
%% Informaci�n de los autores
\title{Din�mica molecular en un cristal binario}

\author{
	\\ \vspace{1cm} Luis Pizarro \and	 
     Pablo Bellino
   }

    \institute{\texttt{\{lpizarro,pbellino\}@cnea.gov.ar}}
\date{{}\\ \vspace{2.5cm} Introduci�n a la Simulaci�n Computacional\\ Universidad de San Mart�n - Centro At�mico Constituyentes \\ Diciembre de 2015}
%%%%%%%%%%%%%%%%%%%%%%%%%%%%%%%%%%%%%%%%%%%%%%%%%%%%%%%%%%%%%%%%%%%%%
\begin{document}

\begin{frame}[plain]
\titlepage
\end{frame}


%% Para incluir una imagen de fondo
%\usebackgroundtemplate{\includegraphics[width=\paperwidth]{fase.png}}

%\frame{\tableofcontents}

%% Resalta la secci�n a la que se pasa
%\AtBeginSection[] {
% \begin{frame}
%    \frametitle{Resumen}
%    \small
%    \tableofcontents[currentsection,hideothersubsections]
%    %\tableofcontents[sectionstyle=show/shaded,subsectionstyle=show/shaded/shaded]
%    \normalsize
% \end{frame}
%}

\section{Introducci�n}

\begin{frame}{Objetivos}
        \begin{itemize}[<+-|alert@+>]
            \item Presentaci�n de los resultados obtenidos durante la puesta en marcha de la CNA U-II (a potencia cero)
        \end{itemize}
\end{frame}

\section{Procedimientos}

\begin{frame}{React�metro}
	\begin{block}{Ecuaciones de la cin�tica puntual}
	\begin{align*}
		\frac{dC_i(t)}{dt} &= \frac{b_i}{\Lambda^*} n(t) - \lambda_i C_i(t) \hspace{1cm} i=1,...,15  \\
		\$(t) &= 1 + \frac{\Lambda^*}{n(t)} \left[\frac{dn(t)}{dt} - \sum_{i=1}^{15} \lambda_i C_i(t)\right] 
	\end{align*}
	\end{block}
    \begin{itemize}
        \item $n(t)$ proporcional a la tasa de detecci�n
        \item Seis grupos de neutrones retardados (Tuttle)
        \item Nueve grupos de fotoneutrones (Keepin)
        \item Discretizaci�n asumiendo evoluci�n lineal de $n(t)$ en un paso temporal. Integraci�n anal�tica de las ecuaciones de los precursores
        \item Paso de discretizaci�n de $0.1\,s$
    \end{itemize}
\end{frame}

\begin{frame}
	\begin{center}
		\begin{tikzpicture}
		\tikzstyle{vector} = [thick,color=red!80!black,>=stealth,->]
		\tikzstyle{atom1} = [circle,shading=ball, ball color=black!60!white, minimum size=5pt]
		\tikzstyle{atom2} = [circle,shading=ball, ball color=orange!50!yellow, minimum size=4pt]
		\begin{axis}[
		view = {40}{20},
		axis lines=none,
		axis equal=true,
		xmin=-0.1,
		xmax=1.1,
		ymin=-0.1,
		ymax=1.1,
		zmin=-0.1,
		zmax=1.1,
		xtick=\empty,
		ytick=\empty,
		ztick=\empty
		]
		\coordinate (A0) at (axis cs:0,0,0);
		\coordinate (A1) at (axis cs:1,0,0);
		\coordinate (A2) at (axis cs:0,1,0);
		\coordinate (A3) at (axis cs:0,0,1);
		
		\coordinate (A4) at (axis cs:1,1,0);
		\coordinate (A5) at (axis cs:1,0,1);
		\coordinate (A6) at (axis cs:0,1,1);
		\coordinate (A7) at (axis cs:1,1,1);
		
		\coordinate (A8) at (axis cs:0.5,0.5,0);
		\coordinate (A9) at (axis cs:0.5,0,0.5);
		\coordinate (A10) at (axis cs:0,0.5,0.5);
		
		\coordinate (A11) at (axis cs:0.5,0.5,1);
		\coordinate (A12) at (axis cs:0.5,1,0.5);
		\coordinate (A13) at (axis cs:1,0.5,0.5);
		
		\draw[black!25] (A0) -- (A8) -- (A4);
		\draw[black!25] (A4) -- (A12) -- (A6);
		\draw[black!25] (A0) -- (A10) -- (A6);
		\draw[black!25] (A1) -- (A8) -- (A2);
		\draw[black!25] (A2) -- (A10) -- (A3);
		\draw[black!25] (A7) -- (A12) -- (A2);
		\draw[black] (A1) -- (A13) -- (A7);
		\draw[black] (A3) -- (A11) -- (A7);
		\draw[black] (A5) -- (A13) -- (A4);
		\draw[black] (A6) -- (A11) -- (A5);
		\draw[black] (A5) -- (A9) -- (A0);
		\draw[black] (A0) -- (A1);
		\draw[black] (A0) -- (A2);
		\draw[black] (A0) -- (A3);
		\draw[black] (A0) -- (A3);
		\draw[black] (A3) -- (A5);
		\draw[black] (A3) -- (A6);
		\draw[black] (A6) -- (A2);
		\draw[black] (A5) -- (A1);
		\draw[black] (A7) -- (A4);
		\draw[black] (A5) -- (A7);
		\draw[black] (A1) -- (A4);
		\draw[black] (A6) -- (A7);
		\draw[black] (A2) -- (A4);
		\fill[atom2] (A12) circle (4pt);
		\fill[atom1] (A0) circle (5pt);
		\fill[atom1] (A1) circle (5pt);
		\fill[atom1] (A2) circle (5pt);
		\fill[atom1] (A3) circle (5pt);
		\fill[atom1] (A4) circle (5pt);
		\fill[atom1] (A5) circle (5pt);
		\fill[atom1] (A6) circle (5pt);
		\fill[atom1] (A7) circle (5pt);
		\fill[atom1] (A8) circle (5pt);
		\fill[atom2] (A9) circle (4pt);
		\fill[atom2] (A10) circle (4pt);
		\fill[atom1] (A11) circle (5pt);
		\fill[atom2] (A13) circle (4pt);
		\draw[vector] (A0) -- (A8);
		\draw[vector] (A0) -- (A10);
		\draw[vector] (A0) -- (A9);
		\end{axis}
%		\node[red,text width=3cm] at (0.0,2) {Vectores de la celda unidad};
		\draw[atom1] (1,0) circle (5pt);
		\node[] at (2.0,0) {�tomo $1$};
		\draw[atom2] (1,-0.5) circle (4pt);
		\node[] at (2.0,-0.5) {�tomo $2$};
		\draw[vector,<->,blue] (A0) -- node[below] {a} (A1);
%		\node[blue] at (4.5,0) {a: par�metro de red};
		\end{tikzpicture}
		
	\end{center}
 \end{frame}


\begin{frame}{Ubicaci�n de los detectores neutr�nicos}
	\begin{columns}[t]
		\hspace{-1cm}
		\begin{column}{0.7\linewidth}
	\begin{center}
		\vspace{-1em}
		%\includegraphics[width=0.85\textwidth]{Nucleo_CNA2.pdf}
	\end{center}
\end{column}
	\hspace{-1.5cm}
	\begin{column}{0.45\linewidth}
		\vspace{-2em}
		\begin{itemize}
			\item Dos detectores de los canales de arranque de la central (BF$_3$)
			\item Una c�mara de ionizaci�n (de tres) especialmente instrumentada para estos ensayos
		\end{itemize}
		\end{column}
	\end{columns}
\end{frame}


\section{Resultados}



\begin{frame}{Calibraci�n del Banco B y del boro}
	\begin{columns}[t]
		\begin{column}{0.45\linewidth}
			\begin{center}
				\small{Se�ales de los detectores}
				%\includegraphics[width=1.0\textwidth]{Debora_03_n.pdf}
			\end{center}
		\end{column}
		\begin{column}{0.45\linewidth}
			\vspace{-1em}
			\begin{center}
				\small{Reactividad estimada (relativa al  valor total del paso)}
				%\includegraphics[width=1.0\textwidth]{Debora_03_r_aatn.pdf}
			\end{center}
		\end{column}
	\end{columns}
\end{frame}

\begin{frame}{Insercci�n del Banco B}
	\begin{columns}[t]
		\begin{column}{0.5\linewidth}
	\begin{center}
		Reactividad de la porci�n introducida del Banco B ($\Delta\$$)
		%\includegraphics[width=1\textwidth]{Debora_03_r_detalle_aatn.pdf}
	\end{center}
		\end{column}
		\begin{column}{0.5\linewidth} 
			\begin{itemize}
			\item Se tiene en cuenta el efecto el tiempo de inserci�n del Banco B mientras se produce el filtrado del moderador
			\item Se extrapola linealmente a la evoluci�n de la reactividad
			\item Se calcula el coeficiente de reactividad del Banco B:
			\begin{block}{}
		\begin{equation*}
			\Gamma_s (s) = \frac{\Delta \$}{\Delta s}
		\end{equation*}
		\end{block}
		\end{itemize}
		\end{column}
	\end{columns}	
\end{frame}


\begin{frame}{Reactividad de barras individuales (caliente)}
	\begin{center}
		Experiencias de ca�da de barras individuales
		\begin{block}{}
\begin{table}[h]
	\begin{center}
%		\captionsetup{width=15cm}
%		\caption{Reactividad de barras individuales en caliente.}
		\label{tb:bcind}
%		\begin{tabular}{cccccc}
%			\toprule
%			\multirow{2}{*}{Barra} & \multirow{2}{*}{$s_i\,[cm]$} & \multirow{2}{*} {$s_f\,[cm]$} & \multicolumn{3}{c}{$\Delta\$/\langle\Delta\$\rangle$}\\
%			\cmidrule(lr){4-6}
%			& & & AI1 & CT1 & CT2\\	\midrule
%			S22 & $98$  & $200$ & $1.03(2)$ & $1.11(5)$ & $0.86(5)$\\ 
%			S23 & $N/R$ & $272$ & $0.89(3)$ & $0.93(9)$ & $1.1862)$\\
%			S11 & $91$  & $251$ & $1.15(2)$ & $1.07(3)$ & $0.78(3)$\\ 	
%			\bottomrule
%		\end{tabular}
	\end{center}
\end{table}
\end{block}
		Las reactividades $\Delta\$$ para cada detector se expresan de forma relativa al promedio $\langle\Delta\$\rangle$ entre los tres detectores.
	\end{center}
\end{frame}



\section{Conclusiones}

\begin{frame}{Conclusiones}
    \begin{itemize}[<+-|alert@+>]
        \item Mediciones y determinaciones independientes de par�metros f�sicos durante la puesta en marcha de la CNA U-II
        \item Implementaci�n del Multi React�metro Digital con tres detectores de forma simult�nea (c�mara de ionizaci�n y canales de arranque) para la estimaci�n de reactividad en los distintos ensayos
    \end{itemize}
\end{frame}

\begin{frame}
	\begin{block}{}
		\centerline {\Huge Muchas gracias.}
	\end{block}
\end{frame}

\appendix

\end{document}
